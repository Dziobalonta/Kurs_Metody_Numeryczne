\documentclass{article}

\usepackage[T1]{fontenc}
\usepackage[polish]{babel}
\usepackage[utf8]{inputenc}
\usepackage{graphicx}
\usepackage{subcaption} 
\usepackage[margin=2cm]{geometry}
\usepackage{listings}
\usepackage{color}
\usepackage{amsmath}
\usepackage{tcolorbox}
\usepackage{systeme}
\usepackage{indentfirst}
\usepackage{dsfont}

\definecolor{red}{RGB}{245, 63, 60}
\definecolor{blue}{RGB}{59, 69, 245}
\definecolor{green}{RGB}{59, 245, 117}
\definecolor{yellow}{RGB}{245, 207, 59}

\title{Raport Zadanie NUM7}
\date{22.12.2023}
\author{Tomasz Dziób}

\begin{document}
  \maketitle
  \newpage
  \section{Wstęp techniczny}
  Poniższy raport dotyczy zadania numerycznego NUM7 z listy 6. Załączone program o nazwie \textit{num7.cpp} został napisany w języku \textit{C++}, oraz korzysta z biblioteki \textit{GNU Plot}. Przed skompilowaniem programu należy zainstalować powyższą bibliotekę.

    \subsection{Jak uruchomić program?}
    Razem z załączonymi plikami znajdziemy \textit{Makefile} który służy do uruchomienia programów.
    \begin{itemize}
      \item \textit{make run} uruchamia program dotyczący podpunktu a wraz ze sprawdzeniem 
  \end{itemize}

  \section{Nakreślenie problemu}
  Interpolację stosuje się do znalezienia funkcji przechodzącej przez znany nam zestaw punktów, które nazywamy węzłami. Różne metody interpolacyjne, mniej lub bardziej dokładnie, przybliżają wartość poszukiwanej funkcji.

  \section{Użyta metoda}
  \subsection{Interpolacja Lagrange'a}
  Jedną z metod interpolacji, oraz tą którą mieliśmy za zadanie omówić, jest \textit{interpolacja wielomianowa}. Polega ona na przeprowadzeniu przez \textit{n-węzłów} wielomianu o stopniu \textit{n-1}.
  Do skontruuowania takiego wielomianu możemy użyć \textit{wzoru Lagrange'a}:
  \begin{center}
    $L_{n}(x) = \sum\limits_{i=0}^{n}f(x_i)l_i(x) = \sum\limits_{i=0}^{n}f(x_i)\prod\limits_{\substack{j=0 \\ j \neq i}}^{n} \frac{x - x_j}{x_i - x_j}$
  \end{center}

  Powyższy wzór służy do obliczania wartości w podanym węźle wielomianu. Za jego pomocą można wyliczyć wartość przybliżanej funkcji w dowolnym punkcie, nie zmieniając jego stopnia. Pozwala to na wykreślenie funkcji bez jawnego poznawania współczynników przy danych potęgach x.
  \begin{table}[h]
    \centering
    \begin{tabular}{|c|c|c|c|c|c|}
        \hline
        $i$ & 0 & 1 & 2 & $\cdots$ & n \\
        \hline
        $x_i$ & $x_0$ & $x_1$ & $x_2$ & $\cdots$ & $x_n$ \\
        \hline
        $f(x_i)$ & $f(x_0)$ & $f(x_1)$ & $f(x_2)$ & $\cdots$ & $f(x_n)$ \\
        \hline
    \end{tabular}
\end{table}

\begin{center}
  \begin{align*}
      l_n(x) = & f(x_0) \frac{(x - x_1)(x - x_2)\cdots(x - x_n)}{(x_0 - x_1)(x_0 - x_2)\cdots(x_0 - x_n)} +\\
      & + f(x_1) \frac{(x - x_0)(x - x_2)\cdots(x - x_n)}{(x_1 - x_0)(x_1 - x_2)(x_1 - x_n)} + \\
      & + f(x_2) \frac{(x - x_0)(x - x_1)\cdots(x - x_n)}{(x_2 - x_0)(x_2 - x_1)\cdots(x_2 - x_n)} + \\
      & \hspace{3cm} \cdots \\
      & + f(x_n) \frac{(x - x_0)(x - x_1)\cdots(x - x_n)}{(x_n - x_0)(x_n - x_1)\cdots(x_n - x_{n-1})}
  \end{align*}
\end{center}

  \textit{Interpolacja wielomianowa} pozwala na obliczenie wartości wielomianu w sposób liniowy(zależne od stopnia wielomianu oraz dokładności w jakiej chcemy wykreślić funkcję).

  \subsection{Dystrybucja węzłów}
  W zadaniu mieliśmy przedstawić powyższy problem dla dystrybucji węzłów:
  \begin{enumerate}
    \item[a)] $x_i = -1 + 2\frac{i}{n+1}$ \qquad $i = (0,1, ..., n)$
    \item[b)] $x_i = cos( -1 + \frac{2i+1}{2(n+1)}\pi)$ \qquad $i = (0,1, ..., n)$
  \end{enumerate}

  \section{Uzyskany wynik}

  \begin{enumerate}
    \item[a)] Przybliżana funkcja $y(x) = \frac{1}{1+50x^2}$ \qquad Dystrybucja węzłów: $x_i = -1 + 2\frac{i}{n+1}$ \\
    \begin{figure}[!ht]
      \begin{minipage}{0.5\textwidth}
        \centering
        \includegraphics[width=\linewidth]{wyniki_a_5.eps}
      \end{minipage}
      \begin{minipage}{0.5\textwidth}
        \centering
        \includegraphics[width=\linewidth]{wyniki_a_10.eps}
      \end{minipage}
    \end{figure}
    \begin{figure}[!ht]
      \begin{minipage}{0.5\textwidth}
        \centering
        \includegraphics[width=\linewidth]{wyniki_a_15.eps}
      \end{minipage}
      \begin{minipage}{0.5\textwidth}
        \centering
        \includegraphics[width=\linewidth]{wyniki_a_25.eps}
      \end{minipage}
    \end{figure}

    \begin{minipage}{\textwidth}
      Na wykresach powyżej jesteśmy w stanie zauważyć jak zmienia się funkcja w zależności od ilości wybranych węzłów. Dla $n = 10$ obserwujemy już zbliżanie się kształtem do pożądanej funkcji. Niestety obecność \textit{efektu Rungego} udziela się już dla wielomianu 14 rzędu co oznacza błąd interpolacji dobranie nieodpowiedniej metody dla tej funkcji.
  \end{minipage}

    \item[b)] Przybliżana funkcja $y(x) = \frac{1}{1+50x^2}$ \qquad Dystrybucja węzłów: $x_i = cos( -1 + \frac{2i+1}{2(n+1)}\pi)$
    \begin{figure}[!ht]
      \begin{minipage}{0.5\textwidth}
        \centering
        \includegraphics[width=\linewidth]{wyniki_b_5.eps}
      \end{minipage}
      \begin{minipage}{0.5\textwidth}
        \centering
        \includegraphics[width=\linewidth]{wyniki_b_10.eps}
      \end{minipage}
    \end{figure}
    \begin{figure}[!ht]
      \begin{minipage}{0.5\textwidth}
        \centering
        \includegraphics[width=\linewidth]{wyniki_b_15.eps}
      \end{minipage}
      \begin{minipage}{0.5\textwidth}
        \centering
        \includegraphics[width=\linewidth]{wyniki_b_25.eps}
      \end{minipage}
    \end{figure}
  \end{enumerate}
  \newpage
  \begin{minipage}{\textwidth}
    Używanie dystrybucji podanej w podpunkcie b) wykazuje się większą odpornością na błędy. Dla $n = 10$ obserwujemy już zbliżanie się kształtem do pożądanej funkcji. Ilość węzłów 15, 25 przybliża w bardzo dokładny sposób ogólny kształt poszukiwanej funkcji. \textit{Efekt Rungego} nie występuje oraz wraz ze wzrostem rzędu wielomianu jesteśmy wstanie zauważyć polepszenie dokładności.
\end{minipage}

  \section{Podsumowanie}
  Skuteczność interpolacji wielomianowej znacząco zależy od funkcji, którą interpolujemy. W sytuacji, gdy zwiększamy liczbę węzłów na końcach przedziału może przyczynić się pojawienia się oscylacji Rungego, co za tym idzie błędu interpolacji. Metoda ta jest szczególnie użyteczna w przypadku funkcji, które przypuszczalnie mają charakter wielomianowy.
\end{document}