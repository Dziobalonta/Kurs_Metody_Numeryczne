\documentclass{article}

\usepackage[T1]{fontenc}
\usepackage[polish]{babel}
\usepackage[utf8]{inputenc}
\usepackage{graphicx}
\usepackage{subcaption} 
\usepackage[margin=2cm]{geometry}
\usepackage{listings}
\usepackage{color}
\usepackage{amsmath}
\usepackage{tcolorbox}
\usepackage{systeme}
\usepackage{indentfirst}
\usepackage{dsfont}

\definecolor{red}{RGB}{245, 63, 60}
\definecolor{blue}{RGB}{59, 69, 245}
\definecolor{green}{RGB}{59, 245, 117}
\definecolor{yellow}{RGB}{245, 207, 59}

\title{Raport Zadanie NUM9}
\date{18.01.2024}
\author{Tomasz Dziób}

\begin{document}
  \maketitle
  \newpage
  \section{Wstęp techniczny}
  Poniższy raport dotyczy zadania numerycznego NUM8 z listy 9. Załączone programy o nazwie \textit{metody.cpp}, \textit{num9\_a.cpp} i \textit{num9\_b.cpp} zostały napisane w języku \textit{C++}, oraz korzystają z bibiotek \textit{GNU Plot}. Przed skompilowaniem programu należy zainstalować powyższe biblioteki.

    \subsection{Jak uruchomić program?}
    Razem z załączonymi plikami znajdziemy \textit{Makefile} który służy do uruchomienia programów.
    \begin{itemize}
      \item \textit{make runA} uruchamia program dotyczący podpunktu a
      \item \textit{make runB} uruchamia program dotyczący podpunktu b
    \end{itemize}

  \section{Nakreślenie problemu}
  Szukanie miejsc zerowych funkcji, zwane również rozwiązywaniem równań nieliniowych, to kluczowy problem w matematyce numerycznej i ma szerokie zastosowanie w wielu dziedzinach nauki. Miejsce zerowe funkcji to punkt, w którym wartość funkcji wynosi zero. Innymi słowy, jest to rozwiązanie równania f(x) = 0.


  Istnieje wiele metod numerycznych do szukania miejsc zerowych, w tym metoda Regula Falsi, metoda Newtona, metoda Siecznych i metoda Bisekcji. Każda z tych metod jest metodą iteracyjną pokazującą inne podejście do odnajdywania pierwiastków zadanej funkcji.

  W zadaniu mieliśmy odnaleźć pierwiastek funkcji $f(x)$ oraz $g(x)$ dla:

  \begin{enumerate}
    \item [\bf{a)}] $f(x) = sin(x) - 0.4$
    \item [\bf{b)}] $g(x) = f(x)^2 = (sin(x) - 0.4)^2$
  \end{enumerate}

  \section{Użyta metoda}
  \subsection{Metoda Bisekcji}
    Metoda ta polega na wzięciu dwóch liczb $x_1$, $x_2$ między którymi podejrzewamy, że znajduje się nasz szukany pierwiastek.

    \begin{enumerate}
      \item[1.] Jako trzeci punkt bierzemy $x_3 = \frac{x_1+x_2}{2} $, czyli środek przedziału.
      \item[2.]Ustalamy, w którym z przedziałów $[x_1, x_3]$, $[x_3, x_2]$ funkcja zmienia znak  $f(x_1) \cdot f(x_2) < 0$, po czym powtarzamy cała procedurę dla tego przedziału.
      \item[3.]Warunkiem stopu jest $|x_{n+1} - x_{n}| \leq \epsilon $.
    \end{enumerate}

    \begin{figure}[!ht]
      \begin{center}
          \includegraphics[scale=0.85]{Bisekcja.eps}
          \caption{Przykład działania Bisekcji}
      \end{center}
  \end{figure}

  \newpage

  \subsection{Metoda \textit{Regula Falsi}}
  Inaczej nazywana \textit{Metoda Fałszywego Położenia}, polega na wzięciu dwóch liczb $x_1$, $x_2$ między którymi podejrzewamy, że znajduje się nasz szukany pierwiastek. Jeżeli znajdziemy dwa punkty, w których znak funkcji jest przeciwny, $f(x_1) \cdot f(x_2) < 0$, jako przyblizenie miejsca zerowego bierzemy punkt przecięcia siecznej przechodzącej przez punkty $(x_1, f(x_1))$, $(x_2, f(x_2))$ z osią OX.

  \begin{enumerate}
    \item[1.] Jako trzeci punkt bierzemy $x_3 = \frac{f(x_1)\cdot x_2-f(x_2)\cdot x_1}{f(x_1)-f(x_2)} $
    \item[2.]Ustalamy, w którym z przedziałów $[x_1, x_3]$, $[x_3, x_2]$ funkcja zmienia znak  $f(x_1) \cdot f(x_2) < 0$, po czym powtarzamy cała procedurę dla tego przedziału.
    \item[3.]Warunkiem stopu jest $|x_{n+1} - x_{n}| \leq \epsilon $.
  \end{enumerate}

  \begin{figure}[!ht]
    \begin{center}
        \includegraphics[scale=0.85]{RegulaFalsi.eps}
        \caption{Przykład działania metody \textit{Regula Falsi}}
    \end{center}
\end{figure}

\newpage

\subsection{Metoda Siecznych}
Bardzo podobna do metody \textit{Regula Falsi} polega na wzięciu dwóch liczb $x_1$, $x_2$, dla których $f(x_1) \neq f(x_2)$. Bez względu na znak $f(x_1) \cdot f(x_2)$, jako przybliżenie miejsca zerowego bierzemy punkt przecięcia siecznej przechodzącej przez punkty $(x_1, f(x_1))$, $(x_2, f(x_2))$ z osią OX.

\begin{enumerate}
  \item[1.] Jako trzeci punkt bierzemy $x_3 = \frac{f(x_1)\cdot x_2-f(x_2)\cdot x_1}{f(x_1)-f(x_2)} $
  \item[2.]Nie patrzymy na znaki przedziałów, zawsze bierzemy dwa ostatnie punkty $[x_3, x_2]$, po czym powtarzamy cała procedurę dla tego przedziału.
  \item[3.]Warunkiem stopu jest $|x_{n+1} - x_{n}| \leq \epsilon $.
\end{enumerate}

\begin{figure}[!ht]
  \begin{center}
      \includegraphics[scale=0.85]{Sieczne.eps}
      \caption{Przykład działania metody Siecznych}
  \end{center}
\end{figure}

Metoda siecznych moze być zbieżna szybciej niż metoda \textit{Regula Falsi}, ale w odróżnieniu od \textit{Regula Falsi} i metody Bisekcji, w niektórych przypadkach zawodzi(nie jest zbieżna do miejsca zerowego).

\subsection{Metoda Newtona}
Metoda ta wymaga znania pochodnej funkcji $f$ aby znaleźć szukane miejsce zerowe. Wybieramy losowy punkt $x_1$ oraz wyliczamy $x_2 = x_1 - \frac{f(x_1)}{f'(x_1)} $. Sprawdzamy warunek stopu oraz jeśli nie osignęliśmy jeszcze zadanej dokładności to do $x_1$ przypisujemy wartość $x_2$.

Jest to metoda której użyjemy w specyficznych przypadkach ze względu wymaganą znajomość pochodnej zadanej funkcji, która nie zawsze jest prosta w obliczeniu.

\begin{enumerate}
  \item[1.]Wyznaczamy pochodną funkcji $f$ oraz wyliczamy $f'(x_1)$
  \item[2.]Jako drugi punkt bierzemy $x_2 = x_1 - \frac{f(x_1)}{f'(x_1)} $
  \item[3.]Przypisujemy do $x_1$ wartość $x_2$ i powtarzamy 
  \item[4.]Warunkiem stopu jest $|x_{n+1} - x_{n}| \leq \epsilon $.
\end{enumerate}

\textbf{Metoda Newtona może byc rozbieżna, zawodzi ona w punktach, w których pochodna znika!}

\newpage

  \section{Uzyskany wynik}
  \begin{enumerate}
    \item[\bf{a)}] Wynik wykonania komendy \textit{make runA}:
    \begin{center}
      \begin{tcolorbox}
        Metoda Bisekcji: 0.411517 \\
        Metoda Regula Falsi: 0.411517 \\
        Metoda Siecznych: 0.411517 \\
        Metoda Newtona: 0.411517
      \end{tcolorbox}
    \end{center}

    \begin{figure}[!ht]
      \begin{center}
          \includegraphics[scale=0.85]{wyniki_a.eps}
          \caption{Wykres dla podpunktu a)}
      \end{center}
    \end{figure}

Jak widzimy dla wszystkich metod udało się osiągnąć zadaną dokładność $\epsilon = 10^{-6}$, 3 z nich w podobnym czasie ok. 5 iteracji, jednak wyraźnie odstającą funkcją jest \textit{Metoda Bisekcji} która potrzebowała aż 20 iteracji.

To jest typowe dla Metody Bisekcji, która jest metodą stabilną, ale wolno zbieżną. W przeciwieństwie do niej, \textit{Metoda Newtona}, \textit{Siecznych} i \textit{Regula Falsi} są zazwyczaj szybsze, ale mogą mieć problemy ze stabilnością w niektórych przypadkach.

Jesteśmy również wstanie zauważyć mniejsze wzrosty i spadki dokładności które są normalnym zachowaniem nie wpływającym na ogólną liniową naturę tej metody.

    \item[\bf{b)}] Wynik wykonania komendy \textit{make runB}:
    \begin{center}
      \begin{tcolorbox}
        Dla g(x):\\
        Metoda Siecznych: 0.411518 \\
        Metoda Newtona: 0.411516 \\
        Dla u(x):\\
        Metoda Siecznych: 0.411517 \\
        Metoda Newtona: 0.411517
      \end{tcolorbox}
    \end{center}
  \end{enumerate}

  \begin{figure}[!ht]
    \begin{minipage}{0.5\textwidth}
      \centering
      \includegraphics[width=\linewidth]{wyniki_b_g(x).eps}
      \caption{Wykres dla dla funkcji g(x)}
    \end{minipage}
    \begin{minipage}{0.5\textwidth}
      \centering
      \includegraphics[width=\linewidth]{wyniki_b_u(x).eps}
      \caption{Wykres dla funkcji u(x)}
    \end{minipage}
  \end{figure}

  \newpage

  Dla funkcji $g(x)$ zastosowanie \textit{metod Bisekcji i Falsi} zakonczyło się niepowodzeniem ze względu na ich kryterium. Wymagają one dwóch punktów w których funkcja przyjmuje ujemne znaki, a $g(x)$ przyjmuje wyłącznie wartości nieujemne.
  Dwie metody które zostały rozwiązały ten problem to \textit{metod Siecznych i Newtona}, jednak poradziły sobie one z tym w ok. 20 iteracji, wynika to z dwukrotności pierwiastka.

  Zastosowanie funkcji $u(x) = \frac{g(x)}{g'(x)}$ poprawiło liczbę potrzebnych iteracji, ponieważ przekształciło oryginalną funkcję $g(x)$ na formę, która jest bardziej przyjazna dla metod takich jak Newtona i Siecznych. Dzielenie $g(x)$ przez jej pochodną sprawia, że nowa funkcja $u(x)$ ma tylko pierwiastki jednokrotne, dzięki czemu metody iteracyjne zbiegają szybciej. Jest to szczególnie korzystne przy funkcjach takich jak $g(x)$, które przyjmują tylko wartości nieujemne i dlatego nie mogą być rozwiązane za pomocą metod takich jak Bisekcja czy Falsi, które wymagają, aby funkcja przyjmowała zarówno wartości dodatnie, jak i ujemne.
  Po zastosowaniu funkcji $u(x)$, obie metody zbiegły znacznie szybciej, w mniej niż 4 iteracjach. To pokazuje, jak efektywne może być zastosowanie funkcji $u(x)$ w przyspieszaniu zbieżności metod iteracyjnych.

  \section{Podsumowanie}
  Ogólnie rzecz biorąc, wyniki te pokazują, że wybór odpowiedniej metody do rozwiązania danego problemu zależy od specyficznych właściwości funkcji, takich jak jej znaki, pierwiastki i znajomość pochodnej. Przekształcenie funkcji na bardziej “przyjazną” formę, taką jak $u(x)$, może znacznie przyspieszyć zbieżność metod iteracyjnych i umożliwić rozwiązanie problemów, które są trudne dla niektórych metod.

  Metoda Bisekcji była najwolniejsza, ale najbardziej stabilna. \textit{Metoda Newtona}, \textit{Siecznych} i \textit{Regula Falsi} były szybsze, ale mogą mieć problemy ze stabilnością. Zastosowanie funkcji $u(x)$ przyspieszyło zbieżność Metody Newtona i Siecznych. Wybór odpowiedniej metody zależy od specyficznych właściwości funkcji.
\end{document}