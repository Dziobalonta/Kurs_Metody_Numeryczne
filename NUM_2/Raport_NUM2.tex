\documentclass{article}

\usepackage[T1]{fontenc}
\usepackage[polish]{babel}
\usepackage[utf8]{inputenc}
\usepackage{graphicx}
\usepackage{subcaption} 
\usepackage[margin=2cm]{geometry}
\usepackage{listings}
\usepackage{color}

\definecolor{dkgreen}{rgb}{0,0.6,0}
\definecolor{gray}{rgb}{0.5,0.5,0.5}
\definecolor{mauve}{rgb}{0.58,0,0.82}

\lstset{frame=tb,
  language=C++,
  aboveskip=3mm,
  belowskip=3mm,
  showstringspaces=false,
  columns=flexible,
  basicstyle={\small\ttfamily},
  numbers=none,
  numberstyle=\tiny\color{gray},
  keywordstyle=\color{blue},
  commentstyle=\color{dkgreen},
  stringstyle=\color{mauve},
  breaklines=true,
  breakatwhitespace=true,
  tabsize=3
}

\title{Raport Zadanie NUM2}
\date{28.10.2023}
\author{Tomasz Dziób}

\begin{document}
  \maketitle
  \newpage
  \section{Wstęp}
  Poniższy raport dotyczy zadania numerycznego NUM2 z listy 1.
  Załączony program o nazwie \textit{num2.cpp} został napisany w języku \textit{C++}, do wykonania
  matematycznych obliczeń został użyty \textit{GSL}. Przed skompilowaniem programu należy
  zainstalować powyższą bibliotekę.
    \subsection{Jak uruchomić program?}
    Razem z załączonymi plikami znajdziemy \textit{Makefile} który służy do
    uruchomienia programu komendą: \textit{make run}.
  \section{Spodziewany wynik}
  Spodziewany wynik rozwiązania równań macierzowych zależy od wartości macierzy A oraz wektora 
  wyrazów wolnych b, a zaburzenie $\Delta$b o małej normie euklidesowej nie powinno znacząco 
  wpływać na rozwiązania, jeśli obliczenia numeryczne są wykonywane w sposób dokładny.

  \section{Użyta metoda}
  Rozwiązanie sprowadza się do obliczenia wartości \textit{y} z wzoru \textit{A $\cdot$ y = b} używając
  wybranych funkcji z biblioteki \textit{GSL}. Wyniki uzyskujemy poprzez zaburzenie wartości 
  wektora b dodając do niego losowe liczby rzędu $10^{-6}$. 

  \section{Uzyskany wynik}
  \begin{lstlisting}
                           === DLA MACIERZY A1 ===
    A1 * y = b
    
    y = [ 0.225084577 -0.006020997 1.841831983 -5.153442774 -0.217622915 ]
    
    A1 * y = (b + Delta_b)
    
    y = [ 837.799675877 -3013.301457583 -390.534586414 797.387978471 996.129372383 ]
    
                            === DLA MACIERZY A2 ===
    A1 * y = b
    
    y = [ 0.577471720 -1.273784582 1.676750084 -4.815794905 0.201563474 ]
    
    A2 * y = (b + Delta_b)
    
    y = [ 0.577471560 -1.273785006 1.676749955 -4.815794644 0.201563572 ]
    \end{lstlisting}
    Spoglądając na wyniki uzyskane po wykonaniu zaburzenia jesteśmy wstanie odrazu zauważyć
    różnicę. Pomimo założeń mówiących, że tak małe wartości nie powinny mieć znaczącego wpływu
    na wynik, okazujemy się być w błedzie. Dla przykładu $A_1$ uzyskujemy wartości liczone w setkach
    a nawet tysiącach, dla konstrastu w drugim nawet nie przekaczają liczby dziesiątek.

    Rozwiązanie $A_1$ z zaburzonym wektorem wyrazów wolnych jest nierealistycznie duże, 
    co sugeruje, że nawet niewielkie zaburzenie $\Delta$b może znacząco wpłynąć na rozwiązanie. 
    Wartości w tym przypadku są znacznie większe niż te w oryginalnym rozwiązaniu.
    Rozwiązanie $A_2$ z zaburzonym wektorem wyrazów wolnych jest tylko nieznacznie zmienione w 
    porównaniu do oryginalnego rozwiązania. To sugeruje, że macierz $A_2$ jest bardziej 
    stabilna względem zaburzeń $\Delta$b w porównaniu do macierzy $A_1$.
  
    Ogólnie rzecz biorąc, wyniki sugerują, że stabilność numeryczna rozwiązań zależy od 
    właściwości macierzy. Macierz $A_2$ wydaje się być bardziej stabilna wobec zaburzeń niż macierz $A_1$. 
    Niemniej jednak, wartości zaburzenia $\Delta$b są na tyle małe (o normie euklidesowej rzędu $10^{-6}$), 
    że na pierwszy rzut oka nie wyglądają jabky miały znaczący wpływ na rozwiązania dla obu macierzy. 

\end{document}
